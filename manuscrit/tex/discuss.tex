\section{Discussion}

En réanalysant le jeu de données d'une précédente étude phylogénétique, il été ici possible de conforter l'histoire évolutive de la sous-section \textit{Erythrodrosum} (\textit{Primula} section \textit{Auricula}) et sa structure génétique au niveau des Alpes. Cette étude a également validé l'hypothèse d'admixture pour un taxon présent dans le massif des Écrins, confirmant de fait  le besoin de révision taxonomique pour l'espèce \textit{P. pedemontana}. Ce passage de la phylogénétique à la génétique des populations sur un même jeu de donnée est permis grâce aux marqueurs utilisés et a autorisé l'analyse des données sous des angles différents. A la différence des marqueurs couramment utilisés en génétique des populations (comme les microsatellites), les SNPs peuvent être génotypés en grand nombre. Ce nombre important de \textit{loci} informatifs permet d'estimer des paramètres génétiques qui auparavant demandaient un très grand nombre d'individus, tout en permettant une forte résolution phylogénétique. Cette complétion entre les deux analyses apporte donc de plus grandes capacités à résoudre des histoires complexes telles que celle de la sous-section \textit{Erythrodrosum}.

\subsection{Structure génétique alpine}

En effet, bien que le traitement des données initiales diffère de la précédente étude \citep{Boucher2016a}, par un fort tri sur les données manquantes, les analyses de structurations de la sous-section \textit{Erythrodrosum} confirment la présence de plusieurs groupes génétiques. Ces derniers sont associés à des espèces aux aires de répartitions bien localisées dans les Alpes. D'un côté un \clade{est-alpin} composé des espèces \textit{P. daonensis} et \textit{P. villosa}, et de l'autre un \clade{Hirsuta} composé de l'espèce homonyme et de \textit{P. pedemontana s.l.}. Cette structure globale de la sous-section était attendue, étant donné qu'il s'agit de plusieurs espèces et que les résultats de phylogénie proposaient une structure à cette échelle. Pour autant il est important de remarquer que la valeur du $F_{st}$ reste faible. Cela peut être du à la radiation récente de la sous-section \textit{Erythrodrosum} \citep{Boucher2016}, qui a pu limiter l'effet de la dérive génétique sur ces massifs isolés. Cette radiation récente permet également d'expliquer le peu de variations morphologiques que présentent les espèces de cette sous-section. C'est d'ailleurs ces fortes ressemblances qui ont longtemps modifiés la taxonomie de ce clade. Une autre hypothèse pour cette faible valeur de $F_{st}$ serait la présence de flux de gènes, homogénéisant les pools d'allèles et luttant contre une dérive génétique.
 
%Une autre hypothèse pourrait être la présence de structures plus fines à celle étudiée comme c'est le cas dans le massif des Apennins \citep{Crema2009}. \todo[inline]{Je ne vois pas trop pourquoi de la structure à fine échelle empêcherait qu'il y ait de la structure à grande échelle.}

Autre résultat de cette étude, l'absence de structure génétique très marquée pour \textit{P. pedemontana s.l.} ajoute un argument en faveur de la révision taxonomique de ce groupe d'espèces. Cette faible structuration pourrait également être causée par l'échelle spatiale et génétique de l'étude. En effet, il ne faut pas oublier que cette structure a été étudiée à l'échelle des Alpes et que par conséquent, la diversité génétique entre les massifs de la Vanoise et des Écrins (intra-\textit{P. pedemontana s.l.}) est forcement beaucoup moins importante qu'entre la Vanoise et l'est des Alpes (\textit{P. daonensis}). Néanmoins, ce résultat est identique à celui de la délimitation d'espèces \citep{Boucher2016a} et soulève la question de la délimitation d'espèces au sein de cette sous-section. Cette question est d'autant plus délicate que ces espèces présentent naturellement des hybrides viables \citep{Schorr2012}.

\subsection{Flux de gènes et glaciations}

Les résultats de cette étude confirment que la lignée de \textit{P. pedemontana} du massif des Écrins présente un signal d'admixture avec \textit{P. hirsuta}, ce qui permet d'affirmer l'existence d'un flux de gène (passé ou actuel) entre ces deux espèces. Cette information est par ailleurs cohérente avec les observations de terrains et explique en partie pourquoi les précédentes études ne peuvent l'assigner comme une espèce à part entière ou comme un hybride. Cependant, ce flux de gène reste difficile à étudier avec ce jeu de données, par le manque d'individus génotypés et par une zone de contact secondaire (passée ou actuelle). Un ré-échantillonnage plus précis des massifs de Vanoise et des Écrins permettrait d'ailleurs de clarifier le statut de cette zone de contact. En effet, l'observation de divers degrés d'admixture affirmerait que le flux de gène est actuel, avec la présence de différentes générations d'hybrides.

Une plus fine compréhension de ce cas d'admixture aiderait à la compréhension de l'histoire évolutive de la flore alpine européenne. En effet, cet événement d'admixture ne représente pas un cas isolé et un certain nombre de cas d'hybrides sont observables entre les différentes espèces du genre \textit{Primula} \citep{Boucher2016a,Boucher2016,Schorr2012,Casazza2012,Kadereit2011}. Cette redondance de cas pourrait trouver son origine dans l'histoire géoclimatique du système alpin, où les cycles glaciaires ont joué un rôle important dans la délimitation géographique des niches écologiques. Ainsi, avec le développement des glaciers alpins, de nombreuses populations se sont vues restreintes dans des refuges (nunataks) \citep{Schneeweiss2011}, et le retrait glaciaire a remis en contact ces populations. Ces contacts secondaires intra-spécifiques ont entraînés un brassage génétique cyclique, aboutissant à une perte d'allèle rares et donc de diversité \citep{Schorr2012}. Ce processus limitant la structurations des différentes espèces a pu jouer un rôle important dans la perméabilité des barrières reproductives inter-spécifiques.

Enfin, bien que la spéciation allopatrique soit la plus commune pour la flore des Alpes \citep{Boucher2016}, la présence de flux de gènes et d'hybrides suggère que ce mode de spéciation requiert un isolement géographique bien plus long (plus de 3 Millions d'années) pour que la dérive génétique puisse effectivement aboutir à deux espèces sœurs. En l'absence de long isolement, ce mode de spéciation serait remplacé par la spéciation hybride, qui demande cependant de reconsidérer la notion d'espèce biologique. La prise en compte de ce mode de spéciation en biologie évolutive questionne finalement notre capacité à comprendre l'histoire de l'évolution, puisque des hybrides provenant de lignées aujourd'hui éteintes ont pu fausser la description d’événements biologiques passés.  

%Un plus grand échantillon permettrais aussi de proposer une datation pour les différentes séparations entre les lignées réparties sur ces différents massifs. Cette datation pourrait être obtenue avec une approche par approximate Bayesian computation (ABC) sur le logiciel DIYABC \citep{Cornuet2014}. Ces datations permettrais d'évaluer l'impact que les cycles glaciaires du quaternaire ont eu sur ces plantes d'altitude et comment l'évolution de leurs niches écologique a impacté leur génétique. De telles connaissances ainsi qu'une meilleure appréciation des tailles effectives de populations guiderais plus sûrement les politiques de protection du milieu alpin face aux changements climatiques à venir. Ces éléments sont également importants pour essayer de comprendre l'avenir de cette lignée et si de futurs contacts secondaires avec les massifs voisins pourraient redistribués des gènes entre ces différentes 'espèces'.

\subsection{Conclusion}

Finalement, la sous-section \textit{Erythrodrosum} présente une structure génétique particulière, façonnée par l'histoire du système alpin européen et son relief favorable aux répartitions allopatriques. Cette histoire aura notamment isolé les lignées sur des massifs éloignés tout en mettant en place d'autres flux de gènes, questionnant ainsi le statut d'espèce pour ces plantes d'altitudes.

%\todo[inline]{
%Le titre de cette section paraissait plus intéressant que celui de la précédente mais au final tu ne parles pas du tout des processus évolutifs et du rôle de l'environnement là-dessus. 
%Tu as l'air obnubilé par le fait que tu n'as pas pu faire tourner DIYABC mais en vérité on s'en fout pas mal, non? Déjà, c'est une méthode parmi d'autres, et en plus de ça une méthode qui fait pas mal d'hypothèses assez restrictives. C'est pas parce que tous les étudiants de Laurence utilisent ce soft qu'il permet de répondre à toutes les questions (cf. les discussions qu'on a eues tous les deux sur la datation relative et bourrée d'hypothèse fortes). 
%Il faut mettre plus de biologie évolutive dans ce rapport: tu avais une jolie question taxonomique à traiter et tu y as répondu de manière assez claire avec tes résultats génétiques qui soutiennent les hypothèses morphologiques, c'est top. Oui, il y a des limites avec ton jeu de données, message reçu et de toute façon tu ne peux rien y faire. Maintenant, qu'est-ce que tes résultats nous disent sur le processus de spéciation, sur l'évolution de la flore alpine ? Si on a une lignée admixée dans les Écrins, pourquoi on n'en a pas ailleurs. Ou alors on en a? D'ailleurs, est-ce qu'il existe des hybrides naturels entre les espèces de la sous-section \textit{Erythrodrosum}? Ils sont nombreux? Qu'est-ce que ça nous dit sur la nature des barrières reproductives entre espèces?
%Au passage, où est passée toute la littérature sur l'évolution de la flore alpine?}
